\documentclass{article}
\usepackage{amsmath}
\usepackage{geometry}
\geometry{a4paper, margin=1in}
\usepackage{amsfonts} % For mathbb if needed later

\title{Computation of MSA-Level Wage Premia}
\author{Project Workflow Automation}
\date{\today}

\begin{document}
\maketitle

\section{Introduction}
This document outlines the methodology used to compute MSA-level wage premia based on ACS data. The goal is to estimate the extent to which wages in different Metropolitan Statistical Areas (MSAs) differ after controlling for individual worker characteristics.

\section{Data Source}
The primary data source for this analysis is a Stata file (`extract19.dta`) containing pooled cross-sections from the American Community Survey (ACS) for the years 2005-2023. Key variables utilized from this dataset include:
\begin{itemize}
    \item \texttt{incwage}: Individual's annual wage and salary income.
    \item \texttt{wkswork1}: Weeks worked last year.
    \item \texttt{uhrswork}: Usual hours worked per week.
    \item \texttt{perwt}: Person weight.
    \item \texttt{age}: Age of the individual.
    \item \texttt{sex}: Sex of the individual.
    \item \texttt{race}: Race of the individual.
    \item \texttt{educ}: Educational attainment of the individual.
    \item \texttt{met2013}: Metropolitan Statistical Area identifier (CBSA 2013 definition).
    \item \texttt{year}: Survey year.
\end{itemize}

\section{Methodology}

\subsection{Hourly Wage Calculation}
First, an hourly wage is computed for each individual $i$ as:
\begin{equation}
    \text{wage\_hr}_i = \frac{\text{incwage}_i}{\text{wkswork1}_i \times \text{uhrswork}_i}
\end{equation}
Individuals with non-positive or non-finite hourly wages are excluded from the analysis. The natural logarithm of the hourly wage is then taken:
\begin{equation}
    \ln(w_i) = \log(\text{wage\_hr}_i)
\end{equation}

\subsection{Regression Model}
For each year $t$ from 2005 to 2023, the following regression model is estimated using Ordinary Least Squares (OLS), weighted by \texttt{perwt}:
\begin{equation}
    \ln(w_{ijmt}) = \beta_0 + \beta_1 \text{age}_{it} + \beta_2 \text{age}_{it}^2 + \mathbf{X}_{it}'\gamma + \alpha_{mt} + \epsilon_{ijt}
    \label{eq:main_regression}
\end{equation}
where:
\begin{itemize}
    \item $w_{ijmt}$ is the hourly wage of individual $i$ in MSA $m$ in year $t$.
    \item $\text{age}_{it}$ and $\text{age}_{it}^2$ control for age in a quadratic form.
    \item $\mathbf{X}_{it}$ is a vector of dummy variables for individual characteristics: sex, race, and educational attainment. $\gamma$ is the corresponding vector of coefficients.
    \item $\alpha_{mt}$ represents the fixed effect for MSA $m$ in year $t$. These are the key parameters of interest.
    \item $\epsilon_{ijt}$ is the error term.
\end{itemize}
The model is specified in `fixest::feols` syntax as:

\texttt{lnw \textasciitilde{} age + I(age	extasciicircum{}2) + factor(sex) + factor(race) + factor(educ) | met2013}

This effectively means that \texttt{met2013} (representing $\alpha_{mt}$) captures the average log-wage difference for MSA $m$ in year $t$ relative to a baseline MSA (omitted by the estimation procedure), after controlling for the specified individual characteristics.

\subsection{Wage Premia}
The estimated fixed effects, $\hat{\alpha}_{mt}$, are interpreted as the skill-adjusted log-wage premia for each MSA $m$ in year $t$. These are denoted as \texttt{fe\_adj\_lnw} in the output data. 

These log-wage premia ($ \texttt{fe\_adj\_lnw}_{mt} $) represent the component of wages attributable to geographic location (MSA) after accounting for differences in the observable skill composition of the workforce across MSAs.

\section{Output}
The primary output containing these premia is \texttt{data/output/msa\_wage\_premia\_2005\_2023\_simplified.csv}. This file includes \texttt{met2013}, \texttt{year}, and \texttt{fe\_adj\_lnw}, among other variables.

\end{document} 